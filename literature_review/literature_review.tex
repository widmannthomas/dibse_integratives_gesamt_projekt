%----------------------------------------------------------------------------------------
%	PACKAGES AND OTHER DOCUMENT CONFIGURATIONS
%----------------------------------------------------------------------------------------

\documentclass{report}

\usepackage[english]{babel}
\usepackage{lipsum}
\usepackage{float}
\usepackage{graphicx}
\usepackage{amsmath}
\usepackage{tikzducks}
\usepackage{pgfplots}
\usepackage[a4paper, total={6in, 8in}]{geometry}
\usepackage[T1]{fontenc}
\usepackage{titlesec, blindtext, color}

\definecolor{gray75}{gray}{0.75}
\newcommand{\hsp}{\hspace{20pt}}
\titleformat{\chapter}[hang]{\Large\bfseries}{\thechapter\hsp\textcolor{gray75}{|}\hsp}{0pt}{\Large\bfseries}

\begin{document}



{\LARGE Literature Review}

% +----------------+
% |    QUELLE 1    |
% +----------------+

{\let\clearpage\relax \chapter{Quelle 1}}
\noindent
\textbf{Autor*innen:} Max Muster\\
\textbf{Referenztyp:} Artikel\\
\textbf{Jahr:} 1843\\
\textbf{Journal / Herausgeber:}\\
\textbf{doi / ISBN:}\\\\
\textbf{Zusammenfassung und Vernetzung:}\\
...\\

% +----------------+
% |    QUELLE 2    |
% +----------------+

{\let\clearpage\relax \chapter{Quelle 2}}
\noindent
\textbf{Autor*innen:} Max Muster\\
\textbf{Referenztyp:} Artikel\\
\textbf{Jahr:} 1843\\
\textbf{Journal / Herausgeber:}\\
\textbf{doi / ISBN:}\\\\
\textbf{Zusammenfassung und Vernetzung:}\\
...\\

% +----------------+
% |    QUELLE 3    |
% +----------------+

{\let\clearpage\relax \chapter{Is WiFi suitable for energy efficient IoT deployments? A performance study}}
\noindent
\textbf{Autor*innen:} Max Muster\\
\textbf{Referenztyp:} Artikel\\
\textbf{Jahr:} 1843\\
\textbf{Journal / Herausgeber:}\\
\textbf{doi / ISBN:}\\\\
\textbf{Zusammenfassung und Vernetzung:}\\
The author of this study analyzed the performance of WiFi in terms of energy efficiency.
Therefor some experiments were made with the ESP-12 which is a WiFi capable MCU.
Specially the impact of battery types, authentication mode and deep sleep cycles on the energy consumption was investigated.
Furthermore, the author described the basic idea behind the different sleep modes of the MCU.
By using the capability of setting the MCU into deep sleep, the author investigated the impact of different
sleep intervals on the power consumption. 
The analyzes of the resulting data show that the establishment and authentication of the connection to the access point takes much more
energy than just sending the data. Other technologies like Zigbee, BLE or LoRaWAN are using a different connection method that requires much less energy.
The analyzes also show that, in addition to establishing the connection, the MCU also requires a lot of energy to transmit and receive data.
This is the main reason why it is so important to keep the wake up (operation) time as short as possible.
\\
\\
The different experiments came to the following results:\\
\paragraph{Battery} (Three different types of batteries were analyzed.):
\begin{itemize}
    \item Alkaline
    \item NiMH
    \item Li-Po
\end{itemize}
The result of this experiment shows that the type of the battery doesn't influence the battery life much.
It is worth mentioning that the different battery types have different discharge curves.

\paragraph{Authentication} (Three different types of authentication were analyzed.):
\begin{itemize}
    \item No authentication
    \item WEP
    \item WPA2
\end{itemize}
The result shows that WPA2 needs the most energy. WEP and no authentication are about the same.
However, WPA2 is strongly recommended for security reasons.

\paragraph{Deep sleep cycle} (Different sleep cycles were analyzed.)
The result shows that there is no impact on the battery life.
\\\\
\textbf{Importance of this paper:}
\\\\
This paper provides the fundamental ideas behind the sleep modes of MCUs.




% +----------------+
% |    QUELLE 4    |
% +----------------+

{\let\clearpage\relax \chapter{RSSI Comparison of ESP8266}}
\noindent
\textbf{Autor*innen:} Max Muster\\
\textbf{Referenztyp:} Artikel\\
\textbf{Jahr:} 1843\\
\textbf{Journal / Herausgeber:}\\
\textbf{doi / ISBN:}\\\\
\textbf{Zusammenfassung und Vernetzung:}\\
In this paper, different ESP8266 modules with different types of antennas have been analyzed.
The type of the antenna affects the RSSI (received signal strength indicator) which is an indicator of the signal strength.
The weaker the signal strength is, the more power is used to transmit data.
That means that the battery life might depend on the signal strength.
\\
The authors of this paper have chosen the ESP8266 module because of it's popularity and it's low price tag.
\\\\
Four different with four different antenna types have been analyzed:\\
\begin{itemize}
    \item \textbf{Model A:} meandered PCB antenna
    \item \textbf{Model B:} inverted-F PCB antenna
    \item \textbf{Model C:} ceramic antenna
    \item \textbf{Model D:} external dipole antenna
\end{itemize}

In the experiment, the RSSI was measured at different distances (20m, 30m and 40m).
As expected, the model with the dipole (model D) performed best. Followed by C, A and B.
\\\\
However, the size of a dipole antenna can be a problem in IoT use cases because one goal is often to keep the device small.
\\\\
\textbf{Importance of this paper:}
\\\\
The knowledge that the power consumption correlates with the signal strength is important to optimize the battery life.
As mentioned, one way to improve the signal strength is to use larger antennas. However, this is not always acceptable.
If an external dipole antenna is not acceptable, a ceramic antenna is preferred over a PCB antenna.


% +----------------+
% |    QUELLE 5    |
% +----------------+

{\let\clearpage\relax \chapter{How DHCP Leases Meet Smart Terminals: Emulation and Modeling}}
\noindent
\textbf{Autor*innen:} Max Muster\\
\textbf{Referenztyp:} Artikel\\
\textbf{Jahr:} 1843\\
\textbf{Journal / Herausgeber:}\\
\textbf{doi / ISBN:}\\\\
\textbf{Zusammenfassung und Vernetzung:}\\
In this paper the basic principle of DHCP (Dynamic Host Configuration Protocol) is described. 
DHCP is used to automatically configure clients in a network. 
It allows to automatically assign IP addresses to new clients in a network as well as configuring the subnetmask, dns, default gateway and many more parameters.
The basic idea of DHCP is, that there is a central device in the network called DHCP server that manages a pool of IP addresses.
Once a device is connected to a network, it sends a DHCP request as a broadcast into the network. 
The DHCP server than answers this requests with the defined network configuration and an IP address together with a lease time.
The lease time defines how long the client is allowed to use the address. After this period of time, the client has to send a new request. 
Normally this happens after the half of the lease time.
The author of this paper analyzes the impact of different lease times on network overhead.


{\let\clearpage\relax \chapter{Assessing the ESP8266 WiFi module for the Internet of Things}}
\noindent
\textbf{Autor*innen:} J. Mesquita, D. Guimarães, C. Pereira, F. Santos and L. Almeida\\
\textbf{Referenztyp:} Conference Paper\\
\textbf{Jahr:} 2018\\
\textbf{Journal / Herausgeber:} IEEE\\
\textbf{doi / ISBN:} 10.1109/ETFA.2018.8502562\\\\
\textbf{Zusammenfassung und Vernetzung:}\\
The authors of this article show the ESP8266 and if it's suitable as a battery-powered IoT device with Wi-Fi connectivity.
They explain the different sleep modes and conduct experiments using them in different contexts.\\\\
The 3 different sleep modes the ESP8266 has, are:\\
\paragraph{Modem-sleep:} Is used for applications that need real-time CPU control. The Wi-Fi module is turned off but every other component remains on. The connection to the AP remains. When waking up, the connection doesn't need to be reestablished and because of that, the latency is minimized.
\paragraph{Light-sleep:} Is used for applications that need less or partially CPU control. The Wi-Fi module and the system clock is turned off. The connection to the AP remains.
\paragraph{Deep-sleep:} This mode has to be called explicitely by the programmer ans is used for applications that process data in large time-intervals. Everything is turned off except the real-time-clock (RTC). When waking up, the connection to the AP has to be reestablished.\\\\
The experiments show that:
\paragraph{Beacon interval and DTIM interval:} 
In Modem-sleep mode, the energy consupmtion is decreased by increasing the beacon interval or the DTIM interval. When the beacon interval is greater than 700ms, the energy consumption in light-sleep mode nearly equals the energy consumption in modem-sleep mode.
\paragraph{Multicast transmissions:}
In scenarios based on multicast transmissions, the energy consumption is way higher than in scenarios without multicast. The ESP8266 wakes up more frequently, is awake much longer and therefore consumes more energy.
\paragraph{Impact of data transmission:}
In one case, 85B are sent every second and in the other case 850B are sent every ten seconds for each power-mode.
\begin{itemize}
        \item Active: No difference in battery lifetime
        \item Modem-sleep: 14\% more lifetime when sending 850B every ten seconds.
        \item Light-sleep: 44\% more lifetime when sending 850B every ten seconds
        \item Deep-sleep: Not possible to send data every second due to connection reestablishement on waking up.
    \end{itemize}
\paragraph{Vernetzung:}
This article is relevant because we want so optimize the battery-lifetime of our ESP8266 and this paper shows us the different sleep-modes and the corresponding impact on the energy-consumption. This is fundamental, when we want to optimize the battery-lifetime of our Wi-Fi sensor.


{\let\clearpage\relax \chapter{Internet of things with ESP8266: build amazing internet of things projects using the ESP8266 Wi-Fi chip}}
\noindent
\textbf{Autor*innen:} Schwartz,
Marco\\
\textbf{Referenztyp:} Book\\
\textbf{Jahr:} 2016\\
\textbf{Journal / Herausgeber:} Packt\\
\textbf{doi / ISBN:} 978-1-78646-802-4\\\\
\textbf{Zusammenfassung und Vernetzung:}\\
The author of this book gives a deep insight on the capabilities of the ESPP8266. He shows how to choose the right version of it and what the advantages and disadvantages of those different versions are. Instructions on how to install the IDE are given. In addition to that, this book offers a wide variety of sample-programs and gives an insight on what the ESP8266 is capable of. Tutorials on how to read data from sensors and how to connect it to the Wi-Fi network are given. It also teaches, how to enable machine-to-machine communication. Example code for the different project are available. The ESP8266 is a low cost microcontroller. It has a 32-bit CPU, so it's a SoC and can work without additional hardware. It can directly connect to the internet with it's built in Wi-Fi module and is fairly easy to program with the Arduino IDE. There are multiple versions of the ESP8266 available. Each of it has more funcionality built into it, which makes it very flexible and suitable for almost every usecase. The ESP8266 is available for about 5\$. Because of the features and the low cost, this device is one of the most popular IoT devices available on the market. To program the ESP8266, the Arduino IDE is used.\\
\paragraph{Vernetzung:} This book is the fundament to work with the ESP8266. It includes everything someone needs to know when working with this IoT device and for us it is the starting point on working with this device. We are making a prototype for this project and this book covers everything we need to know to realize it.


{\let\clearpage\relax \chapter{MQTT based Home Automation System Using
ESP8266}}
\noindent
\textbf{Autor*innen:} R. K. Kodali and S. Soratkal\\
\textbf{Referenztyp:} Conference Paper\\
\textbf{Jahr:} 2016\\
\textbf{Journal / Herausgeber:} IEEE\\
\textbf{doi / ISBN:} 10.1109/R10-HTC.2016.7906845\\\\
\textbf{Zusammenfassung und Vernetzung:}\\
The authors of this article show the need of an energy-saving communication protocol for battery-powered IoT devices like the ESP8266.
They show the functionality of MQTT, how it works and give recommendations on what IDE to use for the ESP8266 as MQTT-client and what programs to use as an MQTT-broker.\\
\begin{itemize}
    \item MQTT works with the publish/subscribe pattern. Every device, that sends and receives messages over MQTT is called an MQTT-client. Such device can either be publisher or subscriber.
    \item A publisher sends data on a given topic to the MQTT-broker and a subscriber subscribes to one or more topics and receives those messages.
    \item The MQTT-broker forwards messages to the right subscribers. Every topic has a name and a level, that depends on the name of the topic.
    \item Publisher and subscriber are isolated from each other.
    \item Establish connection: MQTT-client sends CONNECT packet. Broker responds with CONNACK packet, that contains a code, that informs about the status of the connection.
    \item Subscribe: MQTT-client sends SUBSCRIBE packet with an UTF-8 encoded topic-name. The broker responds with a SUBACK packet.
    \item Publish: MQTT-client sends PUBLISH packet with the topic-name, the message and the QoS.level. MQTT supports 3 QoS-levels
    \item Disconnect: MQTT-client sends DISCONNECT packet and the connection is cut without a response from the broker.
    \item The connection between the client and the broker is cut on a given time interval. To prevent this, the client has to send a PINGREQ packet to the broker to signal, that he is still there. The broker responds with a PINGRESP packet.
    \item To program the ESP8266 as an MQTT-client, the Arduino IDE is recommended and as an MQTT-broker, moquitto is suitable. The URL to the MQTT-broker is the IP-adress of it with the port 1883. To display and send or receive messages on a device it's recommended to use MQTTLens on PC and MyMQTT on android  based smartphones.
\end{itemize}
\paragraph{Vernetzung:}
When using Wi-Fi with IoT, a communication protocol is inevitable. Because the IoT device is battery-powered, a communication protocol with low energy consumption is key. MQTT suits this requirement. This paper describes how to use this protocol and that's needed for us.



{\let\clearpage\relax \chapter{Power Consumption Analysis of a Wi-Fi-based IoT Device}}
\noindent
\textbf{Autor*innen:} Yüksel ME\\
\textbf{Referenztyp:} Journal Article\\
\textbf{Jahr:} 2020\\
\textbf{Journal / Herausgeber:} Electrica\\
\textbf{doi / ISBN:} 10.5152/electrica.2020.19081\\\\
\textbf{Zusammenfassung und Vernetzung:}\\
The authors of this article analyse the power consumption of battery-powered IoT devices and how to optimize it. They use an ESP32 for it, which has the ESP8266 installed. They analyzed options on how to increase the battery-lifetime when using the build-in sleep-modes and changing how the ESP32 communicates with the given network.\\
\paragraph{They came to the following conclusions:}
\begin{itemize}
    \item When deciding either to go on deep-sleep or modem-sleep, the time-interval between two transmissions plays a huge role. When setting the transmission period to 1 minute, the modem-sleep mode consumes less energy than the deep-sleep mode because of the need to wake up and completely reestablish the connection. When setting the interval to 3 minutes and more, the opposite is the case.
    \item Using UDP instead of TCP helps reducing the energy consumption (by about 14\%). But UDP doesn't provide flow control, so packets can be lost and it's not an effective solution.
    \item Using a static IP address rather than using DHCP saves energy because of the overhead, DHCP brings. Longer DHCP leases can also lead to less energy consumption.
    \item The workload on the router and the server can have a huge impact on the connection time and the required energy.  
    \item It's not always the best way to use low-power modes. It depends on the application. 
    \item When uploading data, the data rate can play a role on energy consumption. They show, that at a rate more than 11Mbps, the energy consumption very slowly decreases and it doesn't have a mentionable impact whether there are 140B or 260B to transmit. At a rate less than 11Mbps, the energy consumption fastly decreases and it the size of the packets has a relative high impact.
\end{itemize} 
\paragraph{Vernetzung:}
It's good to know, what sleep-modes we can work with, but it's more important to know, when to use them and how much energy we save. Also it's essential to know, where we can save energy on the network-layer and what uses much energy.





\end{document}