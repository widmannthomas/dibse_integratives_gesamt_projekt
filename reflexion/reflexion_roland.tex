\chapter*{Reflection Roland}


\section*{Introduction}
In these couple of weeks I've gained a lot of high quality experiences in context of academic writing. Our team consisted of Thomas, Arthur and me. 
The biggest learning was, how to start writing an academic paper, finding research questions, conducting the experiments and analyzing the results. 


\section*{Approach}
First I thought we won't be able to make such a good experiment because the time was short and not everybody of us had the desired equipment.
But contrary to my expectations, we have all conducted the experiments with a, in my opinion, big success.
The biggest challenge for me was, to set up the hardware and connect the components together in the right way. 
Thomas helped us a lot with this demanding task. After conducting our experiments, we had to analyze the resulting data. 
This wasn't very tricky, as we've learned how to plot graphs in python from Pascal Schöttle.\\ 
In terms of academic writing, finding the research questions and enough scientific papers was the most difficult task for me, as I had completely no experience in it. 
I've learned that some sort of creativity to find a good research question is required, and it's really not that easy as it sounds. 
A can highly recommend to search for papers other students wrote for a similar topic to get some inspiration. When searching for suitable scientific papers, google scholar was the way to go. 
At first, the search wasn't very difficult, because I've started with the basics to gain an overall knowledge of the topic. 
Then it began to be more difficult, as the topic was restricted more and more.\\ 
A big learning in that field was, that it's not always the best choice to search exactly what you are looking for in terms of keywords typed in the search box. 
Sometimes good and suitable papers arrive when searching for a quite similar topic. \\
This was the first document I wrote with LaTeX. At first, I thought, that's to complex to learn in such a short time, but I was proven wrong. 
With the help of my teammates and some tutorials, it was quite easy to work with it and I consider writing my bachelor thesis with it because of its possibilities and the look of the resulting paper. 
I learned that it's crucial to be good at English understanding because every paper is written in that language which was no problem for me.\\ 
I play with the thought of writing my bachelor thesis in a similar topic, also with microcontrollers and that work has confirmed that this will be the right choice for me. 
I have to say that the conducting of the experiments was very time-consuming because I had no knowledge regarding electricity and how to measure the current, but we supported each other.

\section*{Working atmosphere in the group}
The working atmosphere was very pleasant and everybody worked to achieve the same goal. He helped each other a lot, and it was really a fun project overall.

\section*{Gained knowledge}
Finding a research question. How scientific writing works. How to cite. Working with LaTeX. How to measure the current with an MCU. How to conduct an experiment and evaluate the data.