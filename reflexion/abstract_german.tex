\chapter*{Abstract Deutsch}
Da IoT-Geräte in Bereichen wie dem Gesundheitswesen und Smart-Homes immer beliebter werden und viele dieser IoT-Geräte batteriebetrieben und schwer zu erreichen und zu warten sind, ist es wichtig, dass sie eine lange Batterielebensdauer haben.
Per Definition muss ein IoT-Gerät mit anderen Geräten oder dem Internet verbunden sein. 
Es gibt viele Kommunikationsprotokolle, um diese Aufgabe zu erfüllen, aber in unserem Fall verwenden wir WLAN, weil diese Technologie in fast jedem Haushalt vorhanden ist und viele Smart-Home-Anwendungen diese Technologie nutzen.
Um die Batterielebensdauer solcher IoT-Geräte zu verlängern, gibt es viele Möglichkeiten.
Drei dieser möglichen Optionen werden bewertet und beschrieben. Ihre Auswirkungen auf die Energieeffizienz werden anhand von Experimenten aufgezeigt.
Unsere Ergebnisse zeigen, dass die Verwendung des Deepsleep Modus des Controllers den Stromverbrauch im Vergleich zum Modemsleep Modus um den Faktor $\approx 290$ verbessert.