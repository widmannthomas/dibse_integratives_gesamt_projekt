\chapter*{Reflection Arthur}

\section*{Introduction}
As part of the seminar Integrative Overall Project, Roland Rauter,
Thomas Widmann and I worked on how to improve the energy efficiency of an ESP8266-01 chip in detail.
The goal of our investigation was to analyze what makes the data transfer over Wi-Fi
from an ESP to a server as energy efficient as possible.
Since we all like to experiment,
we decided to work out the results ourselves instate of relying on existing data.


\section*{Approach}
Before we started our experiment, I had to buy necessary parts.
Since I always bought microcontrollers with an integrated USB port and the ESP8266-01
does not have a USB port. Here I could benefit from Thomas's previous knowledge,
because unfortunately only the USB-to-TTL converters work with the CH340 chip under MacOS.
Unfortunately, the converter Thomas used was not available in Italy,
so we searched for an equivalent adapter. When it arrived,
I found that the alternative adapter had the correct specification,
but did not provide enough power to run the ESP with the INA219.
So I used an external power supply. Now that everything worked out,
we benefited from the lectures of Prof. Matthias Janetsche, PhD
to create the test programs and test scenarios. \\
I focused on the data transfer via UDP and TCP.
The next step was to send data and gain data for our comparison.
At the beginning I sent two prepared packets with different sizes via UDP and TCP respectively.
As I had to find out, the results were not usable,
because transmission of data is not done immediately after the send command,
but with a short inconsistent delay.
Then we decided to send 1000 small packets of one byte,
sending data with this method has no delay for some reason.\\
Now,
when we used the send command, it sends immediately,
and it was possible to evaluate the data afterwards.
When all tests were run, and we had the results, it was time for the evaluation.
As shown in the lecture of Prof. Dr. Pascal Schöttle, we used Python for our analysis.
A big advantage of using the Python libraries was that we could generate diagrams programmatic
without other tools. Now we had to write a paper documenting our results.
Since I have very little experience in scientific work, this was the most difficult part for me.
Luckily, at the beginning of our project we were shown how to search for scientifically
relevant sources. I usually searched in forums or on Stack Overflow for answers,
but i have learned, these are not scientifically reliability sources.
In the future I will rather use Google Scholar because it works really well and helpful.
A downside is that not all papers are publicly available.


\section*{Working atmosphere in the group}
Our working atmosphere was very good. Fortunately,
everyone was very motivated and showed great commitment.
Because of that were able to create our own results within the short time frame.
The project was a lot of fun, it is very motivating to try out and learn new things.
\section*{Gained knowledge}
In our project I was able to learn many interesting things about sending data over Wi-Fi
and the energy-saving aspects of microcontrollers.
Previously I have never dealt with these areas of microcontrollers intensively.
In addition, I was able to learn a lot about the scientific gathering
of data and the presentation of results.
Another advantage was the evaluation of the data using Phyton,
as this allowed me to deepen what we have learned in the previous lessons.
