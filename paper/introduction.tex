Internet of Things is an evergrowing technology. It is estimated that in 2025, 75.4 billion IoT devices will be installed around the globe \cite{lucero2016iot}. These devices can be used in a wide range of life applications such as healthcare, transportaion, logistics and even in personal smart home environments as seen in \cite{6803174}.\\  
In order to declare a microcontroller or in general, a programmable device, an IoT device, it needs to communicate in some sort to the internet or to other devices. So in general, IoT is a network of physical objects of all types of devices, not just computers \cite{patel2016internet}. There are numerous communication protocols available to do that job. Some examples are ZigBee, BLE, Z-Wave, NFC and WiFi \cite{8079928}. As seen in \cite{8088226}, WiFi consumes the most energy and is the most complex communication protocol comparing to the others.  
When choosing a suitable connection protocol for a battery powered IoT device, the already available communication protocols and their energy consumption should be considered.\\
In our research paper, we investigate the different methods to extend the battery life of IoT devices with WiFi communication, because it is the most common communication protocol that is available in nearly all households and we want this paper to be in context of smart home applications.\\
A part of a smart home environment can be described as Wireless Sensor Network (WSN). A WSN consists of battery powered stand-alone devices with few sensors, with limited processing power and an interface that allows them to communicate with each other \cite{wsn}. 
An example for WSN in smart home would be a battery powered ESP8266-01 with a temperature sensor that sends data via MQTT to other devices.
MQTT is a light weight Message Queuing Telemetry Transport (MQTT) protocol which can be used perfectly with the ESP8266-01 because of it's light weight and small energy consumption \cite{kodali_mqtt_2016}.\\
In this paper, we evaluate three different approaches to increase the battery life of the ESP8266-01 connected to WiFi.\\
In the next section we describe the theoretical background of the three methods we used. 
In the third section we describe our scientific approach and go into detail, how we conducted our experiments.
In part four of this paper we prepared an overview of our experimental setup and how we did the measurements.
In the fifth section we describe and discuss our results of the experiments and in the last section we present our conclusions on how these experiments impact the battery life.
These three questions are asked and evaluated: Do the different sleep modes have an impact on energy efficiency?; Does the usage of a static IP instead of DHCP improve the energy efficiency?; Is there a difference between sending data over UDP and TCP?
