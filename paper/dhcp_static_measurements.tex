\subsection{DHCP and static IP}
To analyse the power consumption when connecting via DHCP or with a static IP, we monitored the current of an average connection process.\\

\subsubsection{Experimental setup}
The sequence of the test program for DHCP is the same as Fig. \ref{fig:experiment_deep_sleep} shows.
Fig. \ref{fig:experiment_static_ip} shows the sequence when conducting the experiment using a static IP address.
The ESP8266WiFi library provides functionality to set a fixed IP address for our device.\\

\subsubsection{DHCP}
As shown you can see in Fig. \ref{fig:dhcp}, the connection to the WiFi network by using DHCP takes approximately 5.5 seconds and there are three spikes where the ESP8266 takes 140-120 milliamps.
In average, the energy consumption is about 68.9 milliamps and in total, it consumes about 360 maS. 
It's mentionable that in this case, everytime the device connects to the network, it gets a new IP address. This happens because the lease time runs out before the device reconnects again.
This can be optimized by setting longer lease times. When the lease time is long enough and the device doen't get a new IP address, it behaves like it's getting a static IP address on the energy consumption perspective.

\subsubsection{Static IP address}


\begin{figure}[h]
    \centering
    \includegraphics[width = 0.35 \linewidth]{fig/sequence_static_ip.png}
    \caption{Experimental setup for use of a static IP.}
    \label{fig:experiment_static_ip}
\end{figure}
\begin{figure}[h]
    \centering
    \includegraphics[width =\linewidth]{fig/static_ip.png}
    \caption{Experimental setup for use of a static IP.}
    \label{fig:static_ip}
\end{figure}
\begin{figure}[h]
    \centering
    \includegraphics[width =\linewidth]{fig/dhcp.png}
    \caption{Experimental setup for use of DHCP with new IP address every reconnect.}
    \label{fig:dhcp}
\end{figure}