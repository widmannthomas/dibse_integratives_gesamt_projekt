\subsection{DHCP}
In order to connect a device to a IP network, it requires an IP address to communicate.
There are different possibilities to get an IP address.\\ 
One option is to assigne a static IP address to the device. The IP address of the device will never change, it is static. 
On the other hand, the IP address can be aissigned dynamically.
The protocol, that dynamically provides IP addresses to clients in a network is called Dynamic Host Configuration Protocol (DHCP).
It works as follows: When a device wants to connect to the network, it sends a DHCP request as a broadcast message into the network. 
The DHCP server responds to this request with the configured network settings, an IP address and its lease time. \cite{droms1997rfc2131}.
The lease time is defined by the DHCP server. When the time period is running out, the client has to send another request. 
It is common that this happens after the half of the lease time.
If a DHCP server doesn't receive a renew request within the given lease time, the lease of the IP address expires and it can be used for other clients \cite{10.1145/1298306.1298315}.
Choosing the right lease time depends on the given conditions. A huge factor is the size of the network and the grade of mobility of the connected devices. 
When there is a high number of mobile devices connected to the network and they don't stay very long connected, it is recommended that a shorter lease time is used to prevent wasting IP addresses. \cite{khadilkar2007usage}.
But using a shorter lease time leads to a bigger network overhead \cite{li_how_2018}. 
In further sequence, shorter lease times lead to more energy consumption of IoT devices, because the proccess of getting a new valid IP address takes energy.\\
It is proven that using a static IP address rather than using DHCP for address allocation is less energy consuming on IoT devices. 
Longer lease times can also improve the energy efficiency \cite{department_of_computer_engineering_mehmet_akif_ersoy_university_faculty_of_engineering_and_architecture_burdur_turkey_power_2020}. 
This understanding helps us to improve the energy efficiency on the ESP8266-01.



