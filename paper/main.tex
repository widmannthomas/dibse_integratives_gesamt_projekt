\documentclass[conference]{IEEEtran}
\IEEEoverridecommandlockouts
% The preceding line is only needed to identify funding in the first footnote. If that is unneeded, please comment it out.
\usepackage{amsmath,amssymb,amsfonts}
\usepackage{algorithmic}
\usepackage{graphicx}
\usepackage{textcomp}
\usepackage{xcolor}
\usepackage{amsmath}
\usepackage{tabularx}
\usepackage[style=ieee,sortcites=true,maxbibnames=100]{biblatex}
\def\BibTeX{{\rm B\kern-.05em{\sc i\kern-.025em b}\kern-.08em
    T\kern-.1667em\lower.7ex\hbox{E}\kern-.125emX}}


\addbibresource{../bibtex.bib}

\begin{document}

\title{Improving the energy efficiency of battery powered WiFi sensors.\\
{\footnotesize \textsuperscript{*}Using the ESP8266 MCU as an example.}
\thanks{Identify applicable funding agency here. If none, delete this.}
}

\author{\IEEEauthorblockN{Roland Rauter}
\IEEEauthorblockA{\textit{Management Center Innsbruck} \\
\textit{Digital Business and Software Engineering}\\
Innsbruck, Austria \\
roland97@gmx.at}
\and
\IEEEauthorblockN{Arthur Waldner}
\IEEEauthorblockA{\textit{dept. name of organization (of Aff.)} \\
\textit{name of organization (of Aff.)}\\
City, Country \\
email address or ORCID}
\and
\IEEEauthorblockN{Thomas Widmann}
\IEEEauthorblockA{\textit{Management Center Innsbruck} \\
\textit{Digital Business and Software Engineering}\\
Innsbruck, Austria \\
thomas.widmann@icloud.com}
}

\maketitle

\begin{abstract}
As IoT devices gain on popularity in sectors such as healthcare and smart home environments and a lot of these IoT devices are battery powered and difficult to reach and maintain, it's crucial for them to have a long battery life.
Per definition, an IoT device has to connect with other devices or the internet. 
There may be a lot of communication protocols available that do the job but in our case, we are using WiFi because that technology is available in nearly every household and most smart home applications are using this technology.
In order to increase the battery life of such an IoT device, there are a lot of possible options available.
Three of these possible options are evaluated, described. Their impact on the energy efficiency is proved by conducting experiments.


\end{abstract}

\begin{IEEEkeywords}
component, formatting, style, styling, insert
\end{IEEEkeywords}

\section{Introduction}

Internet of Things is an evergrowing technology. It is estimated that in 2025, 75.4 billion IoT devices will be installed around the globe \cite{lucero2016iot}. These devices can be used in a wide range of life applications such as healthcare, transportaion, logistics and even in personal smart home environments as seen in \cite{6803174}.\\  
In order to declare a microcontroller or in general, a programmable device, an IoT device, it needs to communicate in some sort to the internet or to other devices. So in general, IoT is a network of physical objects of all types of devices, not just computers \cite{patel2016internet}. There are numerous communication protocols available to do that job. Some examples are ZigBee, BLE, Z-Wave, NFC and WiFi \cite{8079928}. As seen in \cite{8088226}, WiFi consumes the most energy and is the most complex communication protocol comparing to the others.  
When choosing a suitable connection protocol for a battery powered IoT device, the already available communication protocols and their energy consumption should be considered.\\
In our research paper, we investigate the different methods to extend the battery life of IoT devices with WiFi communication, because it is the most common communication protocol that is available in nearly all households and we want this paper to be in context of smart home applications.\\
A part of a smart home environment can be described as Wireless Sensor Network (WSN). A WSN consists of battery powered stand-alone devices with few sensors, with limited processing power and an interface that allows them to communicate with each other \cite{wsn}. 
An example for WSN in smart home would be a battery powered ESP8266-01 with a temperature sensor that sends data via MQTT to other devices.
MQTT is a light weight Message Queuing Telemetry Transport (MQTT) protocol which can be used perfectly with the ESP8266-01 because of it's light weight and small energy consumption \cite{kodali_mqtt_2016}.\\
In this paper, we evaluate three different approaches to increase the battery life of the ESP8266-01 connected to WiFi.\\
In the next section we describe the theoretical background of the three methods we used. 
In the third section we describe our scientific approach and go into detail, how we conducted our experiments.
In part four of this paper we prepared an overview of our experimental setup and how we did the measurements.
In the fifth section we describe and discuss our results of the experiments and in the last section we present our conclusions on how these experiments impact the battery life.
These three questions are asked and evaluated: Do the different sleep modes have an impact on energy efficiency?; Does the usage of a static IP instead of DHCP improve the energy efficiency?; Is there a difference between sending data over UDP and TCP?


\section{Theoretical background}

\subsection{Sleep modes}
Sleep modes are a very common way to improve the energy efficiency of microcontrollers.
The basic idea is to reduce the power consumption by disabling unused modules on the chip and only power them up when they are required.
In case of the ESP8266-01, the vendor provides three different sleep modes. 
Table \ref{tab_sleep_modes} summarises the capabilities and the theoretical current consumptions.
\cite{mesquita_assessing_2018}

\begin{table}[htbp]
\caption{ESP8266 sleep modes}
\begin{center}
\begin{tabular}{|c|c|c|c|}
\hline
\textbf{Module}&\textbf{Modem sleep}&\textbf{Light sleep}&\textbf{Deep sleep}\\
\hline
\textbf{WiFi} & OFF & OFF & OFF\\
\textbf{AP association} & Connected & Connected & Disconnected\\
\textbf{System clock} & ON & OFF & OFF\\
\textbf{RTC} & ON & ON & ON\\
\textbf{CPU} & ON & Pending & OFF\\
\hline
\textbf{Substrate current} & $15mA$ & $0.4mA$ & $20\mu A$\\
\hline
\end{tabular}
\label{tab_sleep_modes}
\end{center}
\end{table}

\subsubsection{Modem sleep} \label{sec:modem_sleep}
Modem sleep is the default sleep mode of the ESP8266-01 and is recommended for applications that require a real time CPU control. \cite{mesquita_assessing_2018}
By enabling the modem sleep mode, the controller turns off the WiFi modem between the Delivery Traffic Indication Message (DTIM) beacons. 
This improves the power consumption of the system and has the advantage that the system stays connected to the access point (AP).\\
A typical use case is a WiFi controlled light bulb that provides real time light control.\cite{espressif_inc_esp8266_2016}

\subsubsection{Light sleep} \label{sec:light_sleep}
The light sleep mode is similar to the modem sleep mode with the additional improvement that the internal clock is powered off and the CPU is suspended when there are no tasks to execute.\\
According to the datasheet \cite{espressif_inc_esp8266_2016}, it takes less than $3ms$ to switch back into modem sleep mode.\\
This mode can be used when the application needs to stay connected to the access point 
and needs to respond to incoming data. The CPU is powered off when no data arrives.

\subsubsection{Deep sleep} \label{sec:deep_sleep}
For ultra low power applications, the ESP8266-01 provides a deep sleep mode.
In this mode all modules are disabled except the real time clock (RTC) which can be used to wake up the controller periodically.
When the controller is in the deep sleep mode, it can only be woken up by applying a pulse to the reset (RST) pin.
This pulse can either be generated by an external device, for example a motion sensor or the built in RTC module.\\
Another usefull feature is the RTC memory. This kind of memory makes it possible to store data over deep sleep cycles.
It loses the stored data only when the controller is disconnected from the power supply.
A possible use case for that is collecting multiple measurements over time and send them out in a single package.\\
Deep sleep can be used for ultra low power applications that are idle most of the time. 
However, there is the limitation that the system is not reachable from the outside while sleeping. \cite{espressif_inc_esp8266_2016}

\subsection{DHCP}
In order to connect the ESP8266 and in general every device with a Wi-Fi module to the Internet, it needs to have an IP-Address, so it can be identified by the router. 
There exist two possibilities to gain an IP-Address.\\ The first one is to give the device a static IP-Address. As the name suggests, the IP-Address of the device will never change and it will be static. On the other hand, the IP-Address can be given dynamically.
The protocol, that gives dynamically IP-Addresses to the Wi-Fi clients is called DHCP (Dynamic Host Configuration Protocol).
It works as follows: When a device wants to connect to a network, a request to use an address for a time period to the DHCP server is sent. This time period is called "lease" \cite{droms1997rfc2131}. Then the DHCP server responds with the defined network configuration and an IP address.\\
The lease-time is defined by the DHCP server. When the time period is running out, the client has to send another request to the DHCP server. It's common that this happens after the half of the lease time.
If a DHCP server doesn't receive a renew request within the given lease time, the lease of the IP address expires and the client's IP address is used for possible allocation to other clients \cite{10.1145/1298306.1298315}.
Choosing the right lease-time depends on the given conditions. A huge factor is the size of the network and the grade of mobility of the connected devices. When there is a high number of mobile devices connected to the network and they don't stay very long in it. It's recommended that a shorter lease time is chosen, to prevent wasting valuable IP addresses in a sometimes small address pool \cite{khadilkar2007usage}.
But using a shorter lease-time leads to a bigger network overhead \cite{li_how_2018}. In further sequence, shorter lease times lead to more energy consumption of IoT devices, because the proccess of getting a new valid IP address takes energy.\\
It is proven that using a static IP address rather than using DHCP for IP address allocation is less energy consuming on IoT devices. Longer lease times can also improve the energy efficiency \cite{department_of_computer_engineering_mehmet_akif_ersoy_university_faculty_of_engineering_and_architecture_burdur_turkey_power_2020}. This understanding helps us to improve the energy efficiency on the ESP8266.





\subsection{UDP and TCP}
\label{udptcp:sci}
\subsubsection{TCP}
\label{tcp:sci}
Transmission Control Protocol is a connection-oriented network protocol
for sending data over a network.
This means that TCP waits until a connection is established and
then starts transmitting data\cite{postel1981transmission}.
TCP guaratees that data is transmitted to recipiend
in the correct order and without corrupt segments.
As a downside, this creates an enormous overhead compared
to other network protocols\cite{singh2014survey}\newline.
\subsubsection{UDP}
\label{udp:sci}
User Datagram Protocol is a connectionless network protocol
\cite{postel1980user}.
This protocol has no error handling or recovery options for
 the transmission and sends the data continuously.
It is not necessary for the sender that the client 
receives the data.
Compared to TCP, it allows less overhead when transferring data
\cite{singh2014survey}.

\section{Scientific approach}

To answer the questions, we conducted an experiment which is divided into three individual parts.
The indepentent variables are the three different sleep modes, how the WiFi connection is established (DHCP or static IP) and what packets are to sent (UDP or TCP).
We used DHCP with a new MAC-address everytime a connection is established, except when analyzing the behaviour using a static IP address.\\
The dependent variables are the energy consumption in $mA$ and the consumption in amps per second and are therefore ratio scaled.
We used the ESP8266-01 microcontroller as main-device and an FTDI adapter to program it. We used the Arduino IDE with the version 1.8.16 to write and upload the desired sketch.\\
When conducting, what impact the different sleep modes, the ESP8266-01 provides, have, the device will go immediately after the establishment of the WiFi connection over DHCP with a new MAC-address into the given sleep mode.\\
When conducting, if using a static IP instead of DHCP has an impact, we connect the ESP8266-01 either with a static IP or with DHCP and, in case of DHCP, with a new MAC-address to the WiFi network and go immediately after into the deep sleep mode, to better see, what impact that approach has and how long the chosen method takes.\\
When conducting, if there's an impact on energy consumption when using UDP instead of TCP, we connect the ESP8266-01 over DHCP with a new MAC-Address to the WiFi network, send either a TCP or UDP packet, disconnect, wait 2 seconds and repeat this sequence.\\
During the experiment, the WiFi network operates like a normal network in a private household with several devices connected, so it's an experiment with realistic circumstances.\\

To measure the energy consumption, we used an INA219 module which measures the voltage $U_{shunt}$ on a shunt resistor $R_{shunt}$ that is caused by the current $I_{shunt}$ through it.
The current $I_{shunt}$ can finaly be calculated by the ohm's law:
\begin{equation*}
    I = \frac{U}{R}
\end{equation*}
To get a expressive result, we perfomed each experiment 10 times.







\section{Measurements}

The goal of each experiment was, to trace the current consumption of the ESP8266 during different tasks.
For this, we used the following experimental setup:

\begin{figure}[h]
    \centering
    \includegraphics[width = \linewidth]{fig/experimental_setup.png}
    \caption{Setup to measure the current consumption of the ESP8266.}
    \label{fig:experiment_setup}
\end{figure}

The Amperemeter consisted of an INA219 module together with a shunt resistor of $R_{shunt} = 2.2 \Omega$.
With a maximum current consumtion of $I_{max}=170 mA$ \cite{espressif_inc_esp8266_2016}, the maximum voltage drop $U_{shunt_{max}}$ on $R_{shunt}$ is $374mV$.
\begin{align*}
    U_{shunt_{max}} &= I_{max} * R_{shunt}\\
    U_{shunt_{max}} &= 170mA * 2.2 \Omega\\
    U_{shunt_{max}} &= 374mV
\end{align*}

This still leaves $3.3V - 0.374V = 2.926V$ for the ESP8266 which is slightly below the specified voltage range of $3V$ to $3.3V$ \cite{espressif_inc_esp8266_2016}.
However, we had no issues during the experiments that were caused by this lack of voltage.\\
The INA219 measured the current consumption with a frequency of $f_{INA} = 500Hz$.
A higher sampling rate would result into more accurate results but to prove the described power saving concepts it is accurate enough.



\subsection{Sleep modes}

\subsubsection{Experimental setup}
A common use case of a battery powered device is a temperature sensor.
Therefore, we used the following experimental setup:
\textit{The device should measure the temperature with an interval of $T_{measure} = 15min$.
MQTT should be used to publish the measured data to other devices.}

Fig. \ref{fig:experiment_modem_light_sleep} shows the test program for the modem and light sleep.
The step \textit{setRandomMacAddress} makes sure that the ESP8266-01 always gets a new IP address.

\begin{figure}[H]
    \centering
    \includegraphics[width = 0.7 \linewidth]{fig/sequence_modem_light_sleep.png}
    \caption{Experimental setup for modem and light sleep.}
    \label{fig:experiment_modem_light_sleep}
\end{figure}

Fig. \ref{fig:experiment_deep_sleep} describes the experimental setup that we used to perform the tests with the deep sleep mode.
After the \textit{enterDeepSleep} step, the ESP8266-01 sleeps for $15min$ and restarts the sequence again.
The difference between deep sleep and modem / light sleep mode is, that the controller restarts the program every time it wakes up.\\
\begin{figure}[H]
    \centering
    \includegraphics[width = 0.7 \linewidth]{fig/sequence_deep_sleep.png}
    \caption{Experimental setup for deep sleep.}
    \label{fig:experiment_deep_sleep}
\end{figure}

\subsubsection{Modem sleep}
As mentioned in \ref{sec:modem_sleep}, the ESP8266-01 automatically disables the modem when there is no data transmission required.
During transmission the module takes around $80mA$. During modem sleep, $20mA$ are needed.
The modem wakes up every $100ms$ (beacon interval) to keep the connection to the access point established.
Fig. \ref{fig:beacon_interval} shows the beacon interval of $100ms$. 
The same behavior was observed in \cite{montori_is_2017}.

\begin{figure}[H]
    \includegraphics[width = \linewidth]{fig/beacon_interval.png}
    \caption{The power consumption increases to $\approx 80mA$, every time a beacon arrives.}
    \label{fig:beacon_interval}
\end{figure}

\subsubsection{Light sleep}
As described in \ref{sec:light_sleep}, the light sleep mode disables the CPU when no task has to be processed.
When a task has to be processed, the ESP8266-01 switches back into modem sleep.
Fig. \ref{tab_sleep_modes} shows the change from light sleep (green) to modem sleep (red).
It is clear to see that the ESP8266-01 still needs a lot of power when a beacon arrives (every $100ms$).

\begin{figure}[H]
    \includegraphics[width = \linewidth]{fig/light_sleep.png}
    \caption{During the light sleep mode, the power consumption of the ESP8266 is drops to 1.2mA.}
    \label{fig:light_sleep}
\end{figure}

\subsubsection{Deep sleep}
The deep sleep mode is the most efficient sleep mode, as described in \ref{sec:deep_sleep}. 
We achieved a power consumption of only $20 \mu A$. 
However, the ESP8266-01 takes a long time to reconnect to the Wi-Fi network after waking up.
Fig. \ref{fig:deep_sleep} shows the power consumption during the sleep and operating phase.
\begin{figure}[H]
    \includegraphics[width = \linewidth]{fig/deep_sleep.png}
    \caption{During deep sleep mode (green), the power consumption drops to $20 \mu A$}
    \label{fig:deep_sleep}
\end{figure}

\subsection{DHCP and static IP}
To analyse the power consumption when connecting via DHCP or with a static IP, we monitored the current of an average connection process.\\

\subsubsection{Experimental setup}
The sequence of the test program for DHCP is the same as Fig. \ref{fig:experiment_deep_sleep} shows.
Fig. \ref{fig:experiment_static_ip} shows the sequence when conducting the experiment using a static IP address.
The ESP8266WiFi library provides functionality to set a fixed IP address for our device.\\

\subsubsection{DHCP}
As shown you can see in Fig. \ref{fig:dhcp}, the connection to the WiFi network by using DHCP takes approximately 5.5 seconds and there are three spikes where the ESP8266 takes 140-120 milliamps.
In average, the energy consumption is about 68.9 milliamps and in total, it consumes about 360 maS. 
It's mentionable that in this case, everytime the device connects to the network, it gets a new IP address. This happens because the lease time runs out before the device reconnects again.
This can be optimized by setting longer lease times. When the lease time is long enough and the device doen't get a new IP address, it behaves like it's getting a static IP address on the energy consumption perspective.

\subsubsection{Static IP address}


\begin{figure}[h]
    \centering
    \includegraphics[width = 0.35 \linewidth]{fig/sequence_static_ip.png}
    \caption{Experimental setup for use of a static IP.}
    \label{fig:experiment_static_ip}
\end{figure}
\begin{figure}[h]
    \centering
    \includegraphics[width =\linewidth]{fig/static_ip.png}
    \caption{Experimental setup for use of a static IP.}
    \label{fig:static_ip}
\end{figure}
\begin{figure}[h]
    \centering
    \includegraphics[width =\linewidth]{fig/dhcp.png}
    \caption{Experimental setup for use of DHCP with new IP address every reconnect.}
    \label{fig:dhcp}
\end{figure}

\subsection{UDP / TCP}
To analyse the power consumption while sending data to an client we prepared a 
experiment where we send data in a loop to a client over TCP and UDP.\\

\subsubsection{Experimental setup}
To measure the power consumption we had to extend the setup from Fig.\ref{fig:experiment_setup}.
Because the ESP8266 verified its connection to the access point while idle.
To distinguish between the verifying of the connection and the transmitting of
data we log the transmitting process.
In order to obtain only the meaningful parts of the measurements,
we have prepared a trigger via the I/O pins. While the ESP is transmitting,
the PIN 2 goes high.
On the other side the microcontroller logs the lines with 1 when the pin 1 gets current.
\newline
\begin{figure}[h]
\centering
\includegraphics[width = 0.90 \linewidth]{fig/udp_tcp/experimental_setup_udp_tcp.png}
\caption{Setup to measure the current consumption while sending data with the ESP8266.}
\label{fig:experiment_udp_tcp}
\end{figure}
\subsubsection{TCP}
In order to analyze the power consumption when sending data via UDP,
we have prepared a function that transfers data from an ESP8266 to a PC via a WiFi connection.
Whithen we listens to an open port with netcat.
To study the power consumption of TCP,
we will send data to a client over TCP using the ESP8266WiFi library.
In our experiment we prepared random packets to disable the TCP chaching system.
We then processed 100 iterations of the function in Fig.\ref{fig:tcp_uml}
to geather enough values.
\begin{figure}[h]
\centering
\includegraphics[width = 0.7 \linewidth]{fig/udp_tcp/tcp_uml.png}
\caption{process to sending data over TCP}
\label{fig:tcp_uml}
\end{figure}
\newline\newline
As expected, when we send data over TCP, we get a lot of different results.
This is because we have to wait for the recipient response to send the data.
\linebreak\linebreak
\begin{table}[htbp]
\begin{center}
\caption{Results TCP}
\label{tab:table1}
\renewcommand{\arraystretch}{1.8}
\begin{tabular}{l|c|r}
& \textbf{time [ms]} & \textbf{As}\\
\hline
count & 100 & 100\\
mean & 241.570 & 0.021031\\
std & 32.115403 & 0.002591\\
min & 193 & 0.016300\\
25\%  & 219 & 0.019175\\
50\% & 236 & 0.020600\\
75\%  & 257 & 0.022325\\
max & 396 & 0.032600\\
\end{tabular}
\end{center}
\end{table}
\linebreak
If we look at the results, we see that the current fluctuates a lot,
this is because TCP-buil-in system is monitoring the trasmitting 
and waiting for the recipient to receive the packet.\linebreak\linebreak
\begin{figure}[h!]
\centering
\includegraphics[width = 1 \linewidth]{fig/udp_tcp/tcp_s_m.png}
\caption{sending data over TCP}
\label{fig:tcp_s_m}
\end{figure}
\linebreak\linebreak
According these, also the elapsed time is 236$\pm$13.63\% ms
\linebreak
\begin{figure}[h!]
\centering
\includegraphics[width = 0.7 \linewidth]{fig/udp_tcp/tcp_boxplot_time.png}
\caption{elapsed time while sending over TCP}
\label{fig:tcp_boxplot_time}
\end{figure}
\linebreak
In addition, the power consumption is 0.0206$\pm$12.58\% As
\linebreak
\begin{figure}[h!]
\centering
\includegraphics[width = 0.7 \linewidth]{fig/udp_tcp/tcp_boxplot_As.png}
\caption{ampere seconds while sending over TCP}
\label{fig:tcp_boxplot_As}
\end{figure}
\pagebreak
\subsubsection{UDP}
To analyse the power consumption while sending data over UDP we prepared a procedure which
transfer data from an ESP8266 to a generic notbook  via UDP from the WiFiUdp library.
In order to be able to compare with TCP, we create the process in Fig.\ref{fig:udp_uml}
\begin{figure}[h]
\centering
\includegraphics[width = 0.7 \linewidth]{fig/udp_tcp/udp_uml.png}
\caption{process to sending data over UDP}
\label{fig:udp_uml}
\end{figure}
\newline\newline
As expected, when we send data over UDP, we get more uniform values.
\linebreak\linebreak
\begin{table}[htbp]
\begin{center}
\caption{Results UDP}
\label{tab:table1}
\renewcommand{\arraystretch}{1.8}
\begin{tabular}{l|c|r}
& \textbf{time [ms]} & \textbf{As}\\
\hline
count & 100 & 100\\
mean & 95.970 & 0.009227\\
std & 1.058444 & 0.000581\\
min & 94 & 0.008200\\
25\% & 95 & 0.008800\\
50\% & 96 & 0.009100\\
75\% & 97 & 0.009600\\
max & 98 & 0.010900\\
\end{tabular}
\end{center}
\end{table}
\linebreak
If we look at the results, we see that the current is almost stable,
this is because UDP is not waiting for the recipient to receive the packet.
When it starts it sends the Packages simultaneously.\linebreak\linebreak
\begin{figure}[h!]
\centering
\includegraphics[width = 1 \linewidth]{fig/udp_tcp/udp_s_m.png}
\caption{sending data over UDP}
\label{fig:udp_s_m}
\end{figure}
\linebreak\linebreak
According these, also the elapsed time is 96$\pm$1.10\% ms
\linebreak
\begin{figure}[h!]
\centering
\includegraphics[width = 0.7 \linewidth]{fig/udp_tcp/udp_boxplot_time.png}
\caption{elapsed time while sending over UDP}
\label{fig:udp_boxplot_time}
\end{figure}
\linebreak
In addition, the power consumption is 0.0091$\pm$6.38\% As
\linebreak
\begin{figure}[h!]
\centering
\includegraphics[width = 0.7 \linewidth]{fig/udp_tcp/udp_boxplot_As.png}
\caption{ampere seconds while sending over UDP}
\label{fig:udp_boxplot_As}
\end{figure}
\pagebreak

\section{Results} \label{sec_results}

\section{Conclusions}

\subsection{Figures and Tables}

\section*{Acknowledgment}

The preferred spelling of the word ``acknowledgment'' in America is without 
an ``e'' after the ``g''. Avoid the stilted expression ``one of us (R. B. 
G.) thanks $\ldots$''. Instead, try ``R. B. G. thanks$\ldots$''. Put sponsor 
acknowledgments in the unnumbered footnote on the first page.


\printbibliography
\end{document}
