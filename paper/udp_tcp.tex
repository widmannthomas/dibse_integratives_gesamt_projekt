\subsection{UDP and TCP}
\label{udptcp:sci}
\subsubsection{TCP}
\label{tcp:sci}
Transmission Control Protocol (TCP) is a connection oriented network protocol
for sending data over a network.
This means that TCP waits until a connection is established and
then starts transmitting data \cite{postel1981transmission}.
TCP guarantees that data is transmitted to recipiend
in the correct order and without corrupt segments.
As a downside, this creates an enormous overhead compared
to other network protocols \cite{singh2014survey}. \newline
\subsubsection{UDP}
\label{udp:sci}
User Datagram Protocol (UDP) is a connectionless network protocol \cite{postel1980user}.
This protocol has no error handling or recovery options for
the transmission. It sends the data continuously.
It is not important for the sender that the recipient receives the data.
Compared to TCP, it allows less overhead when transferring data \cite{singh2014survey}.