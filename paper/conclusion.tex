In view of the widespread use of IoT devices and WLAN networks, in this paper we have looked at various methods to extend the battery life of WLAN sensors.
To analyze the different methods, we conducted experiments with an ESP8266-01 MCU.
First we analyzed the different sleep modes that are provided by the framework. 
Furthermore, we looked at the impact of DHCP to the power consumption. Therefore, we compared it to the use of a static IP address.
Last but not least, the affects of using UDP instead of TCP was analyzed by us.
\\
We found that the use of a static IP address instead of DHCP improves the power consumption $\approx 20.5\%$.
This is because we save the communication with the DHCP server. Therefore, time and energy is saved.
\\
Using UDP took $\approx 43.9\%$ less energy than TCP. 
The reason for this improvement is, that UDP does not need additional communication between the client and the server.
Nevertheless, you have to be aware that UDP does not guarantee that the data arrives at the endpoint.
\\
The biggest improvement was achieved, by using the deep sleep mode instead of using modem or light sleep.
This made it possible, to reduce the power consumption by $\approx 290\%$.
The reason for this big improvement is, that the deep sleep mode disables nearly every module in the MCU.
\\
Nevertheless, Wi-Fi still needs a lot of energy to communicate with the network and is not optimized for battery powered IoT devices. \cite{spachos_power_2017}

