\subsection{Sleep modes}
Sleep modes are a very common way to improve the energy efficiency of microcontrollers.
The basic idea is to reduce the power consumption by disabling unused modules on the chip and only power them up when they are required.
In case of the ESP8266-01, the vendor provides three different sleep modes. 
Table \ref{tab_sleep_modes} summarizes the capabilities and the theoretical current consumption.
\cite{mesquita_assessing_2018}

\begin{table}[htbp]
\caption{ESP8266 sleep modes}
\begin{center}
\begin{tabular}{|c|c|c|c|}
\hline
\textbf{Module}&\textbf{Modem sleep}&\textbf{Light sleep}&\textbf{Deep sleep}\\
\hline
\textbf{WiFi} & OFF & OFF & OFF\\
\textbf{AP association} & Connected & Connected & Disconnected\\
\textbf{System clock} & ON & OFF & OFF\\
\textbf{RTC} & ON & ON & ON\\
\textbf{CPU} & ON & Pending & OFF\\
\hline
\textbf{Substrate current} & $15mA$ & $0.4mA$ & $20\mu A$\\
\hline
\end{tabular}
\label{tab_sleep_modes}
\end{center}
\end{table}

\subsubsection{Modem sleep} \label{sec:modem_sleep}
Modem sleep is the default sleep mode of the ESP8266-01 and is recommended for applications that require a real time CPU control. \cite{mesquita_assessing_2018}
By enabling the modem sleep mode, the controller turns off the Wi-Fi modem between the Delivery Traffic Indication Message (DTIM) beacons. 
This improves the power consumption of the system and has the advantage that the system stays connected to the access point (AP). 
However, it still requires a lot of power during idle periods because some network traffic, such as broadcast messages, is constantly being received. \cite{gomez_analysing_2011}
A typical use case is a Wi-Fi controlled light bulb that provides real time light control. \cite{espressif_inc_esp8266_2016}

\subsubsection{Light sleep} \label{sec:light_sleep}
The light sleep mode is similar to the modem sleep mode with the additional improvement that the internal clock is powered off, and the CPU is suspended when there are no tasks to execute.\\
According to the data sheet \cite{espressif_inc_esp8266_2016}, it takes less than $3ms$ to switch back into modem sleep mode.\\
This mode can be used when the application needs to stay connected to the access point 
and needs to respond to incoming data. The CPU is powered off when no data arrives.

\subsubsection{Deep sleep} \label{sec:deep_sleep}
For ultra low power applications, the ESP8266-01 provides a deep sleep mode.
In this mode all modules are disabled except the real time clock (RTC) which can be used to wake up the controller periodically.
When the controller is in the deep sleep mode, it can only be woken up by applying a pulse to the reset (RST) pin.
This pulse can either be generated by an external device, for example a motion sensor or the built-in RTC module.\\
Another useful feature is the RTC memory. This kind of memory makes it possible to store data over deep sleep cycles.
It loses the stored data only when the controller is disconnected from the power supply.
A possible use case for that is collecting multiple measurements over time and send them out in a single package.\\
Deep sleep can be used for ultra low power applications that are idle most of the time. 
However, there is the limitation that the system is not reachable from the outside while sleeping. \cite{espressif_inc_esp8266_2016}