\subsection{Sleep modes}
Sleep modes are a common aproach in the IoT area. 
The basic idea is to reduce the power consumption by disabling unused modules in the chip.
In the case of the ESP8266, the vendor provides three different sleep modes. 
\ref{tab_sleep_modes} summarises the capabilities and typical current consumptions.
\cite{mesquita_assessing_2018}

\begin{table}[htbp]
\caption{ESP8266 sleep modes}
\begin{center}
\begin{tabular}{|c|c|c|c|}
\hline
\textbf{Module}&\textbf{Modem sleep}&\textbf{Light sleep}&\textbf{Deep sleep}\\
\hline
\textbf{WiFi} & OFF & OFF & OFF\\
\textbf{AP association} & Connected & Connected & Disconnected\\
\textbf{System clock} & ON & OFF & OFF\\
\textbf{RTC} & ON & ON & ON\\
\textbf{CPU} & ON & Pending & OFF\\
\hline
\textbf{Substrate current} & $15mA$ & $0.4mA$ & $20\mu A$\\
\hline
\end{tabular}
\label{tab_sleep_modes}
\end{center}
\end{table}

\subsubsection{Modem sleep}
Modem sleep is the default sleep mode of the ESP8266 and is recommended for applications that require a real time CPU control. \cite{mesquita_assessing_2018}
By enabling the modem sleep, the ESP8266 will turn off the WiFi modem between the DTIM beacons. 
Importantly, the connection to the AP stays established. This avoids a re-connect over time.
This improves the power consumption of the system and has the advantage that the system stays connected to the AP. \cite{espressif_inc_esp8266_2016}\\
A typical use case is a WiFi controlled light bulb that provides real time light control.

\subsubsection{Light sleep}
The light sleep mode is similar to the modem sleep mode with the additional improvement that the internal clock is powered off and the CPU is suspended.\\
During the light sleep, the CPU does not respond to external interrupt events. 
This has the affect that the ESP8266 needs to be woken up via an external GPIO.
According to the datasheet \cite{espressif_inc_esp8266_2016}, it takes less than $3ms$.\\
This mode can be used when the application needs to stay connected to the access point 
and needs to responde to incoming data. The CPU is powered off when no data receives.

\subsubsection{Deep sleep}
For ultra low power applications, the ESP8266 provides a deep sleep mode.
In this mode are all modules disabled except the RTC which is responsible for a periodic wake up.
The WiFi connection is terminated and the modem is powered off. By applying a pulse to the RST pin of the ESP8266, it can be woken up.
This pulse can be generated periodically by the RTC.\\
The built in RTC memory makes it possible to store data over multiple deep sleep cycles.
This makes it possible to collect multiple measurement over time and send them out in a single package.\\
Deep sleep can be used for ultra low power applications that can be idle most of the time. 
However, there is the limitation that the system is not reachable from the outside at all times. \cite{espressif_inc_esp8266_2016}