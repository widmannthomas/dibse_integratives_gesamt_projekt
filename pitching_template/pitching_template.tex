% !TEX encoding = UTF-8 Unicode
\documentclass[10pt,table]{article}
% landscape document and margins
\usepackage[landscape=true, top=1cm, bottom=.5cm,left=2cm,right=2cm]{geometry}
\usepackage{url,bm,tikz,booktabs,nicefrac,amsmath}
\usepackage{longtable}
\usepackage[ngerman]{babel}
\usepackage[utf8]{inputenc}
%
%%%% some useful commands
% German Gaensefuesschen
\providecommand{\gercom}[1]{\glqq{}#1\grqq{}}

% coloring the background of one cell gray
\providecommand{\lgc}{\cellcolor{lightgray}}


% ------------------------------------------------------------------
\title{Pitching Template}
\date{}
\pagestyle{plain}
% ------------------------------------------------------------------




\begin{document}

\begin{center}
\begin{longtable}{p{4.5cm}p{4cm}p{2.5cm}p{4cm}p{2.5cm}p{3.5cm}}
\toprule
\lgc Pitcher's name & Rauter, Waldner, Widmann & \lgc FoR category & Sensortechnologie & \lgc Date completed & 16.10.2021\\
\midrule
\lgc (A) Working Title & \multicolumn{5}{p{18cm}}{Optimierungsmöglichkeiten der Batterielaufzeit bei WLAN Sensoren im IoT Bereich.} \\[1ex]
\lgc (B) Basic Research Question & \multicolumn{5}{p{18cm}}{Wie kann die Laufzeit eines mit Batterie betriebenen WLAN IoT Sensoren optimiert werden? } \\[1ex]
\lgc (C) Key paper(s) & \multicolumn{5}{p{18cm}}{\cite{mesquita_assessing_2018} \cite{montori_is_2017}} \\[1ex]
 \lgc (D) Motivation/Puzzle & \multicolumn{5}{p{18cm}}{ Für die einfache Nachrüstung eines smarten Sensors ist es von Vorteil, diesen über eine Funktechnologie einzubinden. Um die bestehende WLAN Infrastruktur nutzen zu können, wird diese Technologie bevorzugt.} \\[1ex]
\midrule
\lgc THREE  & \multicolumn{5}{p{18cm}}{\lgc Three core aspects of any empirical research project i.e. the \gercom{IDioTs} guide} \\
\midrule
\lgc  (E) Idea?& \multicolumn{5}{p{18cm}}{Um den Wartungsaufwand der verbauten Sensoren zu minimieren, ist es wichtig, die Batterielaufzeit eines jeden Sensor zu optimieren. Da auf die bereits vorhandene WLAN Infrastruktur aufgebaut werden soll und bekannt ist, dass WLAN für solche Anwendungsfälle aus Engeriegründen weniger gut geeignet ist \cite{thomas_optimizing_2016}, sollen mögliche Optimierungen analysiert und ausgewertet werden.} \\[1ex]
\lgc  (F) Data?& \multicolumn{5}{p{18cm}}{
 	\begin{enumerate}
 		\setlength\itemsep{.1ex}
 		\item Welche Daten möchten Sie (von wem) erheben?
 		\subitem Stromverbrauch pro Messung und Übermittelung und die daraus resultierende Batterielaufzeit
 		\item Wie kommen Sie an die Daten?
 		\subitem Es soll ein Versuchsaufbau aufgebaut werden, welcher die benötigten Daten liefern soll.
 		\item Qualität/ Reliabilität der Daten? Mögliche Probleme?
 		\subitem Messfehler, Fertigungstoleranzen
 	\end{enumerate}} \\[1ex]
\lgc (G) Tools? & \multicolumn{5}{p{18cm}}{\textbf{Versuchsaufbau:} Espresif ESP8266, DHT11 (Temperatursensor), Arduino IDE, Multimeter } \\[1ex]
\midrule
\lgc  TWO & \multicolumn{5}{p{18cm}}{\lgc Two key questions} \\
\midrule
\lgc (H) What’s New?& \multicolumn{5}{p{18cm}}{ Es werden verschiedene Optimierungsmöglichkeiten bzgl. Energieverbrauch für den ESP8266 und deren Auswirkungen aufgelistet. } \\[1ex]
\lgc (I) So What? & \multicolumn{5}{p{18cm}}{ Durch die Optimierung des Energieverbrauches wird der Wartungsaufwand und die damit verbundenen Kosten minimiert. Ebenso können dadurch Resourcenschonende Systeme entwickelt werden. } \\[1ex]
\midrule
\lgc ONE & \multicolumn{5}{p{18cm}}{\lgc One bottom line} \\
\midrule
\lgc  (J) Contribution& \multicolumn{5}{p{18cm}}{Eine Auflistung an Optimierungsmöglichkeiten und deren Auswirkungen.} \\[1ex]
 \lgc (K) Other Considerations& \multicolumn{5}{p{18cm}}{Die Entwicklung eines funktionsfähigen Versuchsaufbau und dessen Auswertung.} \\[1ex]
\bottomrule   
\end{longtable}
\end{center}

%----------------------------------------------------------------------------------------
%	CITE
%----------------------------------------------------------------------------------------

\bibliographystyle{IEEEtran}
\bibliography{cite}

\end{document}



