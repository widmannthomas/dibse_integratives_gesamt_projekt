% !TEX encoding = UTF-8 Unicode
\documentclass[10pt,table]{article}
% landscape document and margins
\usepackage[landscape=true, top=1cm, bottom=.5cm,left=2cm,right=2cm]{geometry}
\usepackage{url,bm,tikz,booktabs,nicefrac,amsmath}
\usepackage{longtable}
\usepackage[ngerman]{babel}
\usepackage[utf8]{inputenc}
%
%%%% some useful commands
% German Gaensefuesschen
\providecommand{\gercom}[1]{\glqq{}#1\grqq{}}

% coloring the background of one cell gray
\providecommand{\lgc}{\cellcolor{lightgray}}


% ------------------------------------------------------------------
\title{Pitching Template}
\date{}
\pagestyle{plain}
% ------------------------------------------------------------------




\begin{document}

\begin{center}
\begin{longtable}{p{4.5cm}p{4cm}p{2.5cm}p{4cm}p{2.5cm}p{3.5cm}}
\toprule
\lgc Pitcher's name & A. U. Thor & \lgc FoR category & Themengebiet & \lgc Date completed & Abgabedatum\\
\midrule
\lgc (A) Working Title & \multicolumn{5}{p{18cm}}{Prägnanter/informativer Titel} \\[1ex]
\lgc (B) Basic Research Question & \multicolumn{5}{p{18cm}}{In einem Satz: Was ist Ihre Forschungsfrage?} \\[1ex]
\lgc (C) Key paper(s) & \multicolumn{5}{p{18cm}}{Identifizieren Sie die für Ihre Arbeit zentrale Literatur (Angabe Literaturbeleg). Im Idealfall finden Sie ein Paper, maximal jedoch 3 Papers. Im Idealfall von \gercom{Gurus} aus dem Feld, aus einem aktuellen gut bewerteten Journal} \\[1ex]
 \lgc (D) Motivation/Puzzle & \multicolumn{5}{p{18cm}}{Beschreiben Sie Ihre akademische Motivation in einem kurzen Absatz (maximal 100 Wörter) – beispielsweise ein \gercom{Puzzle} oder \gercom{Rätsel} das Sie lösen möchten.
 } \\[1ex]
\midrule
\lgc THREE  & \multicolumn{5}{p{18cm}}{\lgc Three core aspects of any empirical research project i.e. the \gercom{IDioTs} guide} \\
\midrule
\lgc  (E) Idea?& \multicolumn{5}{p{18cm}}{Identifizieren Sie die \gercom{Kernidee}, die dieses Forschungsthema antreibt. Wenn möglich, nennen Sie die zentrale Hypothese(n). Nennen Sie die zentrale abhängige Variable und die wichtigste(n) unabhängige(n) Variable(n). Gibt es theoretische Widersprüche o.ä., die ausgenutzt werden können? Im Falle von nicht quantitativen Untersuchungen: Welche Belege aus der Literatur existieren bereits? Aus welchem Grund kann eine alternative Betrachtung spannend sein?} \\[1ex]
\lgc  (F) Data?& \multicolumn{5}{p{18cm}}{
 	\begin{enumerate}
 		\setlength\itemsep{.1ex}
 		\item Welche Daten möchten Sie (von wem) erheben? 
 		\item Welche Sample-Größe schlagen Sie vor?
 		\item Wie kommen Sie an die Daten? Fragebogen? Interviews? Experimente? Timeframe?
 		\item Qualität/ Reliabilität der Daten? Mögliche Probleme?
 	\end{enumerate}} \\[1ex]
\lgc (G) Tools? & \multicolumn{5}{p{18cm}}{Grundlegender empirischer Rahmen? Statistische Methoden? Interviews?} \\[1ex]
\midrule
\lgc  TWO & \multicolumn{5}{p{18cm}}{\lgc Two key questions} \\
\midrule
\lgc (H) What’s New?& \multicolumn{5}{p{18cm}}{Was ist neu? Die Idee/Daten/Methoden? } \\[1ex]
\lgc (I) So What? & \multicolumn{5}{p{18cm}}{Warum ist die Antwort wichtig/interessant?} \\[1ex]
\midrule
\lgc ONE & \multicolumn{5}{p{18cm}}{\lgc One bottom line} \\
\midrule
\lgc  (J) Contribution& \multicolumn{5}{p{18cm}}{In einem Satz: Was wird das zentrale Ergebnis der Arbeit sein?} \\[1ex]
 \lgc (K) Other Considerations& \multicolumn{5}{p{18cm}}{Gibt es irgendwelche Herausforderungen oder eventuell Risiken? Wenn ja welche? Gibt es ethische Bedenken?} \\[1ex]
\bottomrule   
\end{longtable}
\end{center}

\end{document}



